%* installation (deps, build/install procedure, making module/ml work)
%* configuration (either via ./configure at build time, or via
%   $LMOD_FOO), most important configure options
%* setting up the system spider cache: configuring Lmod, updating spider
%  cache, difference between system/user spider cache
%* + all (other) aspects mentioned on slide #26 in this slide deck:
%   explained + example :)
%* what settarg is all about

\documentclass{beamer}

\usetheme[headernav]{TACC} %%Drop the 'headernav' if you don't like
                           %%the stuff at the top of your slide

\usepackage{amsmath,amssymb,amsthm}
\usepackage{alltt}
\usepackage{graphicx}

\title{Lmod \& XALT}


\author{Robert McLay}
\institute{The Texas Advanced Computing Center}

\date{Sept. 3, 2015}  %% Use this if you want to fix the date in
                        %% stone rather than use \today

\newcommand{\bfnabla}{\mbox{\boldmath$\nabla$}}
\newcommand{\laplacian}[1]{\bfnabla^2 #1}
\newcommand{\grad}[1]{\bfnabla #1}
\newcommand{\tgrad}[1]{\bfnabla^T #1}
\newcommand{\dvg}[1]{\bfnabla \cdot #1}
\newcommand{\curl}[1]{\bfnabla \times #1}
\newcommand{\lap}[1]{\bfnabla^2 #1}

\begin{document}

\begin{frame}
  \titlepage
\end{frame}

\section{Lmod}

\begin{frame}{Lmod \& XALT Outline}
  \begin{itemize}
    \item Introduction
    \item Lmod: Our Env. Module System
    \item XALT: Track program \& library usage
    \item Live Demos
  \end{itemize}
\end{frame}

\begin{frame}{Lmod: Our Module System}
  \begin{itemize}
    \item What are Modules?
    \item A clever idea (not invented here!)
    \item User Load Only the Packages they want
    \item Developers can easily switch compilers/mpi
  \end{itemize}
\end{frame}


\begin{frame}{Why invent Y.A. Module System?}
  \begin{itemize}
    \item Many thing right with the Original.
    \item But it was designed before multi-compilers/MPI
    \item Sites make it work but with herculean efforts
    \item Lmod can make this work easily.
  \end{itemize}
\end{frame}


\begin{frame}{Why You Might Want To Switch}
  \begin{itemize}
    \item Active Development;  Frequent Releases; Bug fixes.
    \item Vibrant Community
    \item It is used from Norway to Isreal to New Zealand from Stanford to MIT to NASA
    \item Enjoy many capabilities w/o changing a single module file
    \item Debian and Fedora packages available
    \item Many more advantages when you're ready
  \end{itemize}
\end{frame}

\begin{frame}{Features requiring no changes to modulefiles}
  \begin{itemize}
    \item Reads TCL modulefiles directly (Cray modules supported)
    \item User default and named collections of modules
    \item Module cache system: Faster avail, spider, etc
    \item Tracking module usage
    \item A few edge cases where Env. Modules and Lmod differ
  \end{itemize}
\end{frame}

\begin{frame}{Features of Lmod with small changes to modulefiles}
  \begin{itemize}
    \item Family function: Prevent users from loading two compilers at
      the same time (experts can override)
    \item Properties: (MIC-aware, Beta, etc)
    \item Sticky modules
    \item ...
  \end{itemize}
\end{frame}

\begin{frame}{Lmod supports a software hierarchy}
  \begin{itemize}
    \item Lmod supports flat layout of modules
    \item Some really cool features if you have a software hierarchy
      \begin{itemize}
        \item Protecting users from mismatched modules
        \item Auto Swapping of Compiler and MPI dependent modules
      \end{itemize}
    \item When you are ready, it will be there
  \end{itemize}
\end{frame}

\section{A Transition Strategy}

\begin{frame}{A Transition Strategy}
  \begin{itemize}
    \item Install Lua and Lmod in your account
    \item Staff \& Friendly Opt-in Testing
    \item Deploy to your users with an Opt-out choice
    \item Some users can run TCL/C modules (a.k.a. Tmod)
    \item Others can run Lmod
    \item No single user can run both at the same time!
    \item Transistion doc: lmod.readthedocs.org
  \end{itemize}
\end{frame}

\begin{frame}{Personal Testing of Lmod}
  \begin{itemize}
    \item Install it in your account
      \begin{itemize}
        \item Take current list and remove in reverse order to find
          required modules
        \item LMOD\_SYSTEM\_DEFAULT\_MODULES=mod1:mod2:mod3
        \item \texttt{\bf \$ module purge;}
        \item Redefine module command to use Lmod
        \item \texttt{\bf \$ module restore;}
      \end{itemize}
    \item I normally run a personal copy of Lmod in all my accounts
      (including \texttt{\bf darter}, a Cray XC-30)
    \item Test, build a spider cache, module avail
    \item Module collections, ...
  \end{itemize}
\end{frame}

\begin{frame}{Staff \& Friendly User Testing}
  \begin{itemize}
    \item Install Lmod in system location.
    \item Install \emph{/etc/profile.d/z00\_lmod.sh} to redefine the
      \texttt{\bf module} cmd.
    \item Load system  default modules (if any) after previous step
    \item Users who have a file named \texttt{$\sim$/.lmod} use Lmod
    \item At TACC we did this for 6 months
  \end{itemize}
\end{frame}

\begin{frame}{Deploy}
  \begin{itemize}
    \item When you are ready, change \emph{/etc/profile.d/z00\_lmod.sh}
    \item Users can opt-out
    \item We supported this for another 6 months
    \item We broke Env. Module support when we added the family function
    \item Both transitions generated very few tickets (2 + 2)
  \end{itemize}
\end{frame}

\begin{frame}{Lmod Features}
  \begin{itemize}
    \item Support for a Hierarchical Module layout
    \item Module spider: find all modules
    \item Caching system for rapid avail and spider
    \item Support for Properties
    \item Module collections, output to stdout, proper version sorting
    \item Reads TCL modulefiles directly (Cray modules supported)
    \item And so much more...
  \end{itemize}
\end{frame}

\begin{frame}{Lmod Demo}
  \begin{itemize}
    \item ml purge; ml intel/15.0.2 impi phdf5; ml
    \item ml swap impi mvapich2; ml
    \item ml av
    \item ml spider phdf5;
    \item ml spider phdf5/1.8.14;
  \end{itemize}
\end{frame}

\section{XALT}

\begin{frame}{XALT: What runs on the system}
  \begin{itemize}
    \item A NSF Funded project: PI: Mark Fahey and Robert McLay
    \item A Census of what programs and libraries are run
    \item Running at TACC, NICS, U. Florida, KAUST, ...
    \item Integrates with TACC-Stats.
  \end{itemize}
\end{frame}

\begin{frame}{Design Goals}
  \begin{itemize}
    \item Be extremely light-weight
    \item Provide provenance data: How?
    \item How many use a library or application?
    \item Collect Data into a Database for analysis.
  \end{itemize}
\end{frame}

\begin{frame}{Design: Linker }
  \begin{itemize}
    \item The linker (ld) wrapper intercepts the user link line.
      \begin{itemize}
        \item Generate assembly code: key-value pairs
        \item Capture tracemap output from ld
        \item Transmit collected data in *.json format
      \end{itemize}
  \end{itemize}
\end{frame}

\begin{frame}{Design: Launcher}
  \begin{itemize}
    \item Program Launcher: mpirun, aprun, ibrun ...
      \begin{itemize}
        \item Find Executable by parsing command
        \item Collect executable info, shared libraries, env.
        \item Transmit collected data in *.json format
      \end{itemize}
    \item In future: This will be handled by the linker phase
  \end{itemize}
\end{frame}

\begin{frame}{Design: Transmission to DB}
  \begin{itemize}
    \item File: collect nightly
    \item Syslog: Use Syslog filtering
    \item Direct to DB.
  \end{itemize}
\end{frame}

\begin{frame}{Lmod to XALT connection}
  \begin{itemize}
    \item Lmod spider walks entire module tree.
    \item Can build A Reverse Map from paths to modules
    \item Can map program \& libraries to modules.
    \item /opt/apps/i15/mv2\_2\_1/phdf5/1.8.14/lib/libhdf5.so.9
      $\Rightarrow$ phdf5/1.8.14(intel/15.02:mvapich2/2.1)
  \end{itemize}
\end{frame}

\begin{frame}{Lmod: Priority Path}
  \begin{itemize}
    \item Fixed Job Launcher: ibrun, aprun
    \item Variable Launchers: mpirun, mpiexec
    \item Priority Path: \\
      \texttt{prepend\_path\{"PATH", "/opt/apps/xalt/1.0/bin", priority=100\}}
  \end{itemize}
\end{frame}

\begin{frame}{XALT Demo}
  \begin{itemize}
    \item Show modules hierarchy
    \item ml --raw show xalt
    \item type -a {ld,mpirun}
    \item Build programs
    \item Run tests
    \item Run utf8 program
    \item Show database results
  \end{itemize}
\end{frame}


\section{Conclusion}
\begin{frame}{Conclusion}
  \begin{itemize}
    \item Lmod:
      \begin{itemize}
        \item Source: github.com/TACC/lmod.git, lmod.sf.net
        \item Documentation: lmod.readthedocs.org
      \end{itemize}
    \item XALT:
      \begin{itemize}
        \item Source: github.com/Fahey-McLay/xalt.git, xalt.sf.net
        \item Documentation: doc/*.pdf
      \end{itemize}
  \end{itemize}
\end{frame}

\end{document}
